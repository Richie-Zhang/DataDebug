% \textbf{Software bugs: well-studied, effective tools.}
In many computational tasks, correctness is a primary concern. Most
work in the programming language community over the past decades has
focused on ways to discover whether the program performing the computation is
correct. Techniques to reduce program errors range from
testing~\cite{unittesting,fuzztesting} and runtime
assertions~\cite{samanderansthing,others}, to dynamic and static
analysis tools that can discover a wide range of
bugs~\cite{valgrind,dawsonthing,otherpcmemberfoo}. Using these
approaches and tools greatly increases the ability of programmers to
find errors and reduce their impact, contributing to improving overall
code quality.

% Data errors, not so much.
However, a program is just one part of a computation. When its input
has errors, the result of the computation is likely to not be
correct. Data errors can arise in a number of ways, including mistakes
in data entry or corrupted data sources. Unlike programs, data cannot
easily be checked for correctness.

% \textbf{Data errors really important in data-intensive programming environments.}

While data errors pose a threat to the correctness of any computation,
they are especially problematic in data-intensive programming
environments like databases, spreadsheets, and certain scientific
computations (e.g., data analysis using R~\cite{FIXME}). In these settings,
data correctness can be as important as program correctness (``garbage
in, garbage out''). The results produced by the
computations---queries, formulas, charts, and other analyses---can end
up being rendered invalid by data errors. These errors can be costly:
for example, spreadsheet errors have led to losses of millions of
dollars~\cite{FIXME}.

%%%%  Things to add in citations: %%%%
% TransAlta took a $24 million charge -- copy and paste error
% http://www.skillsportal.co.za/page/training/articles/512049-Spreadsheet-errors-can-be-a-major-cost-to-your-business#.UJbfX2l250s
% http://www.flintshirechronicle.co.uk/flintshire-news/local-flintshire-news/2010/02/18/flintshire-county-council-school-cash-blunder-down-to-spreadsheet-error-51352-25856321/
% http://binnenland.nieuws.nl/566978


% cite approximate computation stuff?

By contrast with the proliferation of tools at a programmer's disposal
to find program errors, few tools exist to help find data errors. Part
of the problem is that, unlike program errors, it is more difficult to
decide whether any given data element is an error or not. For example,
the number \texttt{132} might be correct, or it could be a
transposition error from \texttt{123}. More insidiously, a misplaced
or omitted decimal point could change a data item by orders of
magnitude. Unfortunately, manual data auditing to find this kind of
mistake is both onerous and difficult to scale up to even the moderate
size of data in spreadsheets.


% \textbf{Existing approaches don't really work.}

Existing approaches to finding data errors include
statistical \emph{outlier detection} and \emph{data cleaning}. Outlier
detection can be used to find errors only when the input data follows
a known distribution (e.g., Gaussian). Automatic identification of
data distributions is error-prone and can give rise to an excessive
number of false positives. Data cleaning primarily copes with errors
in databases via cross-validation with ground truth data, which may
not exist.

\subsection*{Contributions}

This paper presents an approach to locating likely data errors that we
call \emph{data debugging}. Data debugging leverages the fact that in
data-intensive programming environment like spreadsheets or databases,
data and programs (e.g., queries or formulas) are intertwined.

%, data debugging reframes the problem to find data
%whose presence has an unusually large impact on the computation as a
%whole.

Data debugging uses static and dynamic analysis to guide a statistical
process that isolate data whose impact dramatically affects the final
results of a computation. It first builds a dependency graph of
the program, and then systematically measures the effect of re-running
the computation after replacing data items with other items drawn
randomly from the same input set.

By calling attention to data that has an unusually large impact on the
final computation, data debugging provides insights into both the data
and the computation and can reveal errors. Since it is impossible to
know \emph{a priori} whether data are erroneous or not, data debugging
does the next best thing: locating data where an error would have the
most impact. Intuitively, data that has an unusually high impact on
the final result is either very important or it is wrong. By
contrast, data that is wrong but whose presence or absence has little
impact on the final result does not merit special attention.

This paper presents the first tool for data debugging that operates in
the context of spreadsheets. This system, called \checkcell{}, works
as a plug-in for Microsoft Excel, though its principles are broadly
applicable. \checkcell{} builds a dependency graph of the entire
computation represented by a spreadsheet, where outputs and
intermediate nodes include formulas and charts, and where data cells
form the leaves. \checkcell{} then performs a statistical perturbation
analysis of the effect of inclusion or exclusion of individual cells,
measuring their impact on the spreadsheet's outputs by recalculation
on the perturbed inputs. \checkcell{} employs kernel density
estimation to evaluate which outputs are unusual and how unusual they
are. The result is a ranking by influence of data whose effect crosses
a threshold of unusualness. In the user interface, \checkcell{} colors
the cells containing this data in shades proportionally to their
impact: the more impact, the brighter the highlighting.

\checkcell{} is efficient: it operates in time linear in the number
of data elements, which is optimal. The current prototype is untuned
but analysis time is generally low, taking less than a minute to run
on spreadsheets containing thousands of cells. A user study verifies
the hypothesis that data debugging's approach of identifying data with
unusual impacts is effective at locating errors. With \checkcell{}'s
help, users were able to find XX\% of injected errors, while users
without \checkcell{} were only able to find YY\% of errors.

%We have developed a Microsoft Excel extension, or add-in, written in
%C\#, which uses cross-validation techniques and perturbation analysis
%to look for numerical values that appear suspicious.  Values that
%appear to be potential errors are highlighted proportionally to the
%likeliness that they are incorrect, as judged by our analysis -- the
%more likely a value is to be wrong, the brighter the highlighting.
