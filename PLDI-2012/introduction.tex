Software bugs are a well-known problem, and substantial research
effort has been expended developing tools which encourage the writing
of correct programs.  However, most programs operate on data supplied
by users, and even a perfectly good program would produce the wrong
results if it is given incorrect input.  Error checking of inputs is
generally limited to ensuring the correct data type and formatting as
required by the program.  However, data is important and there should
be better ways of checking its correctness.

One could imagine a tool which automatically audits data, similar to
the ones which check for typographical errors in text.  However,
numerical data is much more unstructured than text.  There are no
rules, such as grammar, governing how numerical data should look, and
no predefined set of allowable values such as the set of words in a
language.  There may, however, exist some ways of narrowing the scope
of the problem to make it more manageable.

If we look at how data is most commonly stored nowadays, we see that
it is in spreadsheets, such as Excel files or databases, and this can
provide some degree of structure for approaching the problem.  For
that reason, we attempted to create a tool which would automatically
detect numerical errors in spreadsheets.  We have developed a
Microsoft Excel extension, or add-in, written in C\#, which uses
cross-validation techniques and perturbation analysis to look for
numerical values that appear suspicious.  Values that appear to be
potential errors are highlighted proportionally to the likeliness that
they are incorrect, as judged by our analysis -- the more likely a
value is to be wrong, the brighter the highlighting.
