% anomaly issue: we want to compare the effect of making 30 (K) random choices to 
% systematic exploration of the space (as in, Sum_{j != i}{|effect(a_j)-effect(a_i)} / N).
% compare false positives (we have non-anomaly but by bad luck, we exchange with anomalous impacts),
% and false negatives (we have an anomaly but by bad luck, we exchange with anomalous impacts).

This section presents an analysis of several aspects of data
debugging: its asymptotic runtime, the effectiveness of its sampling
approach to measure impact, and a justification of the use of the
normal distribution to score and select outliers from the impacts.

\subsection{Asymptotic Analysis}

Data debugging operates in several phases: computing the dependence
graph, performing impact analysis, and then ranking impacts. Runtime
depends on the following parameters: the number of data items ($n$),
the number of formulas ($f$), and the number of inputs ($i$). In
spreadsheets, the number of inputs equals the number of data items,
but in other contexts like databases, inputs correspond to fields, so
$i \ll n$. Since each formula and data item must be examined at least
once to compute the dependence graph and to measure impact,
respectively, runtime for data debugging must be 
$\Omega(n+f)$.

The cost of building the dependence graph varies depending on the
structure of the computation. It has a worst-case runtime of
$O((i*f)^2)$), quadratic in the total number of inputs and formulas,
because it is theoretically possible for each formula to depend on
every input and other formula. However, this kind of pathological
structure is atypical. Dependence graphs normally form a tree or a
forest of trees. In this case, the cost of constructing the dependence
graph becomes linear in the number of inputs and formulas, or
$O(i+f)$.

Impact analysis dominates the costs of constructing the dependence
graph and ranking impacts, since it requires recalculation of the
computations affected by changes in the data.

A naive implementation of impact analysis that checked the impact of
each data item by systematically replacing it with every other item in
the same range would require $O(n^2)$ time. Worse, each of these
iterations requires potentially costly recalculations. Such an
approach would make data debugging unusable in practice.
 
By using a fixed number of random selections, data debugging keeps the
total number of recalculations to $\Theta(n)$, linear in the number of
data items, which is optimal. Section~\ref{sec:sampling_effectiveness}
explains why this strategy approximates the effect of a complete
examination of other inputs, while minimizing execution time.

Finally, ranking impacts involves only two linear passes over the
impacts to compute absolute z-scores, so it also operates in (optimal)
linear time in the number of impacts.

\subsection{Sampling Effectiveness}
\label{sec:sampling_effectiveness}

A complete strategy for measuring the impact of any data item $i$ is to
systematically measure the impact of replacing it with every other
item $j$ in the same range:

\begin{equation}
\mbox{impact}_i = \frac{\sum_{j \neq i}^{N}{|\mbox{result}_i-\mbox{result}_j|}}{N}
\end{equation}

\noindent
This approach, as mentioned above, would take $O(n^2)$ time and so is
inefficient for large data sets.

Instead, data debugging uses a sampling-based strategy that chooses a
fixed number of samples $K$ (e.g., $K = 30$), with replacement, from
the data range:

\begin{equation}
\mbox{estimated\_impact}_i = \frac{\sum_{j \in \{\mbox{sample}\}}^{N}{|\mbox{result}_i-\mbox{result}_j|}}{K}
\end{equation}

\noindent
There are two cases where this procedure could fail; below, we bound their likelihood.

\begin{itemize}
\item \textbf{False negatives.} The item under consideration has a
  large actual impact, but the sampling procedure repeatedly
  chooses other items with similarly anomalous impacts, so the item under
  consideration appears to have no impact (the anomalous values cancel out).
\item \textbf{False positives.} The item under consideration does
  not have an significant impact, but the sampling procedure
  repeatedly chooses items that have a large impact. Now, the item
  under consideration appears to have an anomalous impact, though it actually does not.
\end{itemize}

In both cases, the key is to ensure that the estimated impact does not
repeatedly sample data items with high impact. We state our theorem informally:

\begin{theorem}
The estimated impact is a close approximation to the actual impact, because it minimizes the risk of false positives and false negatives.
\end{theorem}

Data debuggging considers a data item to have a high actual impact
only when that impact is at least two standard deviations from the
mean impact in that range. Because we assume a normal distribution of
impacts (see below), the number of items with such impact is by
definition less than 5\%.

One can thus view the sampling procedure as a series of Bernoulli
trials, repeatedly flipping a biased coin with a probability $p =
0.05$ of choosing a value whose impact is anomalous (heads) and a
probability $q = 0.95$ of choosing a value with a non-anomalous impact
(tails). For $n$ coin flips and probability $p$ of heads, the expected
number of times that the sampling procedure will choose an anomalous
value is just $np$, or 1.5.

Because the Poisson distribution is an excellent approximation as long
as $n$ is at least 20 and $p$ is no more than $0.05$, it is convenient
to use it to bound the likelihood of choosing a larger number of
high-impact values than $x$, where $x > 1.5$ and $\lambda = np$:

\begin{equation}
Pr[X \geq x] \leq \frac{e^{-\lambda}(e \lambda)^x}{x^x}
\end{equation}

We can use this equation to show that it is highly unlikely that the
sampling procedure would accidentally choose a large number of
high-impact values. For example, the probability of choosing one-third
of the high-impact values from a sample of size $K = 30$ is less than
$3/100,000$. Therefore, the estimated impact computed by sampling is
nearly as effective (reducing both false positives and false
negatives) as computing the actual impact.

\punt{
\paragraph{Normal distribution of impacts.}

Justifies use of trimming (only reporting things that are two standard
deviations)--high confidence that they are outliers.  Average change,
independent (random), same distribution (don't exclude same cell
swap). Use 30 samples.

\paragraph{Probability of missing important data.}

30 samples. Odds of missing something important are vanishingly small.
Cite Feller. Limitations: threshold functions.

\paragraph{Limitations}

\begin{itemize}
\item {\bf Array formulas.}
\item {\bf Tables.}
\item {\bf Control flow / HLOOKUP / VLOOKUP.}
\item {\bf Outputs that change data type.}
\item {\bf Macros / side-effects.}
\end{itemize}
}


\subsection{Impact Outlier Analysis}

Once all impact scores have been computed, only those data whose
impacts cross some threshold of anomalousness should be
reported. We currently treat the impact scores as if they fit a normal
distribution, and only report data whose scores place them more
than two standard deviations away from the mean. In a normal
distribution, that is just under 0.05\% of the population; in other
words, this corresponds to a 95\% confidence level that these are
anomalies.

In the absence of knowledge of the distribution of impact scores, using
the normal distribution is in general a conservative approach. That is, it
is likely to report few false negatives, at the risk of introducing some false
positives. This fact derives from two particular characteristics of the
normal distribution: its low \emph{skewness} and \emph{kurtosis}.

The normal distribution has zero skewness, where skew is the
distribution around the mean; in other words, it is perfectly
symmetric. Any asymmetric distribution by definition has a greater
number of points either to the left or to the right of the mean. By
choosing outliers from the tails of the normal, \checkcell{} also
includes the skewed tails of any other distribution.

In addition, the normal has either low or zero kurtosis, which
indicates the peakedness of the curve and heaviness of the tail;
whether the normal has a kurtosis of 0 or 3 depends on the definition
of kurtosis. Counting outliers from the perspective of the normal
distribution is generally conservative, because it includes all
distributions with heavier tails (those with positive
kurtosis). Because distributions with negative kurtosis generally have
small tails, they also tend to have few outliers by definition. Thus,
failing to report outliers for distributions with negative kurtosis is
generally safe.

One exception worth noting and a limitation of this approach is
symmetric bimodal distributions, where the outlier values are not only
in the tail but also centered around the mean. Using the normal as a
reference would not uncover such outliers.

% we just need to report outliers in the impact. assuming the impacts
% are normal is a conservative approach: the normal has 0 skewness
% (skew = distribution around the mean -- normal is symmetric, so 0
% skew) and low (either 0 or 3) kurtosis, depending on your definition
% of kurtosis. Every non-normal distribution is by definition more skewed and most
% distributions have a higher kurtosis (heavier tails), so we may
% overmark outliers. We won't find outliers in distributions with
% negative kurtosis, but those are super weird (no tails -- they drop
% below the x-axis at some point on either side), so it's hard to
% argue that they have outliers at all.
