% The formula here is: discuss related work in one facet; end with a contrast with the current work.

\paragraph{Data Cleaning.}
Most past work on locating or removing errors in data has focused
on \emph{data cleaning} (also known as \emph{data scrubbing}
and \emph{cleansing}~\cite{DBLP:journals/debu/RahmD00}) in database
systems. Standard approaches include statistical outlier
analysis~\cite{FIXME}, interpolation to fill in missing data (e.g.,
with averages)~\cite{FIXME}, and using cross-correlation with other
tables to correct or locate errors~\cite{FIXME}.

% Section~\ref{FIXME} shows that statistical outlier analysis often produces unacceptably large numbers of false positives. 

% SURVEY! http://www.dbis.informatik.hu-berlin.de/dbisold/research/bioinformatics/papers/data_cleansing.html

A number of approaches have been developed that allow data cleaning to
be expressed programmatically or applied interactively. Programmatic
approaches include AJAX, which expresses a data cleaning program as a
DAG of transformations from input to
output~\cite{Galhardas:2000:AED:342009.336568}. Data Auditor applies
rules and target relations entered by a
programmer~\cite{Golab:2010:DAE:1920841.1921060}. A similar
domain-specific approach has been employed for data streams to smooth
data temporally and isolate it spatially~\cite{1617508}. Potter's
Wheel, by Raman and Hellerstein, is an interactive tool that lets
users visualize and apply data cleansing
transformations~\cite{Raman:2001:PWI:645927.672045}. 
Luebbers et al. describe an interactive data mining approach based on
machine learning that builds decision trees from databases. It marks
deviations from derived logical rules (e.g., ``$\mbox{BRV} =
404 \Rightarrow \mbox{GBM} = 901$'') as errors to be examined by a
data quality engineer~\cite{Luebbers:2003:SDD:1315451.1315499}.

Unlike these approaches, data debugging operates entirely
automatically (without the need for programmer-supplied rules or
latent logical relations in data) by measuring the interaction of data
with the programs that operate on them.
 

\paragraph{Spreadsheet Errors.}
Spreadsheets have been one of the most prominent computer applications
since the creation of the first spreadsheet application, VisiCalc, in
1979. The most widely used spreadsheet application today is Microsoft
Excel. Excel provides rudimentary error detection including errors in
formula entry like division by zero, a reference to a non-existient
formula or cell, invalid numerical arguments, or accidental mixing of
text and numbers.
% http://office.microsoft.com/en-us/excel-help/find-and-correct-errors-in-formulas-HP010066255.aspx
Excel also checks for inconsistency with adjacent formulas and other
structural errors, which it highlights with a ``squiggly'' underline. In addition, Excel provides a formula auditor, which lets spreadsheet view dependencies flowing into and out of a particular formula.
% http://office.microsoft.com/en-us/excel-help/use-error-checking-to-correct-common-errors-in-formulas-HA010342331.aspx

Pre-existing error-detection work: adding type
systems~\cite{DBLP:conf/kbse/AhmadAGK03}. Dataflow
testing~\cite{fisher2006scaling}. Other testing tools (Rothermel et
al.\
~\cite{rothermel1998you,rothermel2001methodology,Carver:2006:EET:1159733.1159775}). XeLda:
checks if formulas process values with incorrect units or if derived
units clash with unit
annotations~\cite{Antoniu:2004:VUC:998675.999448}.  More type system
stuff:~\cite{Erwig:2005:AGM:1062455.1062494}. Using labels and
structural clues (especially unit-of-measurement
errors)~\cite{Chambers:2010:RSL:1860134.1860346}. Much of this work is
complementary and orthogonal to \checkcell{}, which works with
standard, unannotated spreadsheets and focuses on unusual interactions of
data with formulas.


% What is Jaffry et al\. ~\cite{DBLP:journals/corr/abs-0803-1748}?
