% The formula here is: discuss related work in one facet; end with a contrast with the current work.

\paragraph{Data Cleaning.}
Most past work on locating or removing errors in data has focused
on \emph{data cleaning} (also known as \emph{data scrubbing}
and \emph{cleansing}~\cite{DBLP:journals/debu/RahmD00}) in database
systems. Standard approaches include statistical outlier
analysis~\cite{FIXME}, interpolation to fill in missing data (e.g.,
with averages)~\cite{FIXME}, and using cross-correlation with other
tables to correct or locate errors~\cite{FIXME}. None of these
approaches leverage the interaction of data with programs that operate
on them, as data debugging does.

% Section~\ref{FIXME} shows that statistical outlier analysis often produces unacceptably large numbers of false positives. 

% SURVEY! http://www.dbis.informatik.hu-berlin.de/dbisold/research/bioinformatics/papers/data_cleansing.html

A number of approaches have been developed that allow data cleaning to
be expressed programmatically or applied interactively. AJAX expresses
a data cleaning program as a DAG of transformations from input to
output~\cite{Galhardas:2000:AED:342009.336568}. Data Auditor applies
rules and target relations entered by the
programmer~cite{Golab:2010:DAE:1920841.1921060}. A similar
domain-specific approach has been employed for data streams, primarily
aimed at smoothing the data temporally and isolating data
spatially~\cite{1617508}. Raman and Hellerstein's Potter's Wheel is an
interactive tool that allows users to visualize and apply data
cleansing transformations~\cite{Raman:2001:PWI:645927.672045}. By
contrast with this work, \checkcell{} and data debugging in general
operate automatically by discovering data that have an unusually large
impact on program output, although they do not attempt to correct data
errors.

\paragraph{Data Auditing.}
Luebbers et al. VLDB 2003~\cite{Luebbers:2003:SDD:1315451.1315499}.

\paragraph{Spreadsheet Errors.}
Spreadsheets have been one of the most prominent computer applications
since the creation of VisiCalc in 1979~\cite{FIXME}. The most widely
used spreadsheet application today is Microsoft Excel~\cite{}. Excel
indicates errors in formula entry like division by zero, a reference
to a non-existient formula or cell, invalid numerical arguments, or
accidental mixing of text and numbers.
% http://office.microsoft.com/en-us/excel-help/find-and-correct-errors-in-formulas-HP010066255.aspx
Excel can also check for inconsistency with adjacent formulas and other structural errors, which it
highlights with a ``squiggly'' underline.
% http://office.microsoft.com/en-us/excel-help/use-error-checking-to-correct-common-errors-in-formulas-HA010342331.aspx

Pre-existing error-detection work: adding type
systems~\cite{DBLP:conf/kbse/AhmadAGK03}. Dataflow
testing~\cite{fisher2006scaling}. Other testing tools (Rothermel et
al.\
~\cite{rothermel1998you,rothermel2001methodology,Carver:2006:EET:1159733.1159775}). XeLda:
checks if formulas process values with incorrect units or if derived
units clash with unit
annotations~\cite{Antoniu:2004:VUC:998675.999448}.  More type system
stuff:~\cite{Erwig:2005:AGM:1062455.1062494}. Using labels and
structural clues (especially unit-of-measurement
errors)~\cite{Chambers:2010:RSL:1860134.1860346}. Much of this work is
complementary and orthogonal to \checkcell{}, which works with
standard, unannotated spreadsheets and focuses on unusual interactions of
data with formulas.


% What is Jaffry et al\. ~\cite{DBLP:journals/corr/abs-0803-1748}?
