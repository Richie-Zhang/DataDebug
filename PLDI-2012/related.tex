The most commonly used software for editing spreadsheets, Microsoft
Excel, has close to no built-in error checking. Aside from some type
verification, the user simply has to be careful in their work, which
is frequently not the case [15].

Spreadsheet languages allow users to specify their own mechanisms for
error checking, which can be effective if done properly. Users can
define assertions with filters or IF statements making sure values
contained in cells or regions fit certain criteria. Combined with the
AVERAGE and STDEV (standard deviation) functions, one is able to
remove the outliers in a range by checking if their values fall within
x standard deviations from the average. Such approaches to error
reduction are rarely put to use, however, and automatic auditing
methods are desired.

As a result of this necessity, significant efforts have been spent by
researchers trying to address the issue of spreadsheet
validity. However, the multitude of usage scenarios and the absence of
any predefined structure in spreadsheets contribute greatly to the
difficulty of this problem. Because users are free to arrange
spreadsheet data in any way they please (vertically, horizontally, or
in any arbitrary fashion), creating a tool that would work in the
general case is a major undertaking. While per-company auditing
solutions can be developed with relative ease because a standardized
format can be agreed upon from the start, adopting such standards on a
large scale is near impossible. Nevertheless there have been proposed
general methodologies for spreadsheet organization [10]. In the same
way regular programming can be improved by paradigms such as
object-orientation and modularity, spreadsheet development can also
benefit from similar ideas. However, as is the case with other types
of programming, following these principles typically reduces errors,
but does not eliminate them altogether. Also, while they may help
individuals learn how to manage spreadsheets in ways that will reduce
the likelihood of errors, these strategies are not widely used and
therefore do not provide any significant assistance in the creation of
automatic auditing tools.  Some work has also gone into classifying
the types of errors that may appear in spreadsheets. While this
taxonomy is useful in pointing out the kinds of errors that may need
to be addressed and how often they show up in practice, the ways in
which they can be dealt with is not established [12].

Some semi-automatic tools have been created which require that users
specify the range of cells they want to audit, and possibly include
additional ranges to be used for reference [4]. While this is in fact
a useful tool which is able to extract relationships between cells
which may not necessarily be numeric, unassisted discovery of regions
is the natural step forward towards automatic error detection
[6]. This is what our approach aims to achieve when dealing with
numerical values.

It is worth noting that errors in computational logic are close to
impossible to pinpoint because one cannot know the user's intent.
