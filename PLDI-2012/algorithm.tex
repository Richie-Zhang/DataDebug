\punt{
\subsection{Building a Dependence Graph}

A single spreadsheet typically contains multiple different sets of
data, and each of them will have its own properties.  For instance,
the range of numerical values may differ greatly between sets within
the same spreadsheet.  This is important because when we are looking
for errors, values have to be analyzed within their proper context,
and values from different datasets will not be logically compatible.
Therefore in order to be able to detect errors, one has to start by
identifying the contiguous sets of data within the spreadsheet.
Usually a person is able to do this with relative ease by looking at
the visual layout of a spreadsheet and using clues such as names of
columns or rows to break apart the spreadsheet into its composing
sets.  However, spreadsheets provide an extremely flexible way of
storing data.  There are hardly any organizational restrictions to the
user and this lack of constraints makes it quite difficult to
automatically extract the sets of related data by examining the
layout.  Even though many spreadsheets are laid out in some structured
way, there are certainly some that are counterintuitive, and the
variability among layouts is immense.

Regardless of how a spreadsheet is laid out, however, its elements
remain the same.  That is, all the data items, graphs, and formulas
can be moved around without affecting the content.  The underlying
logic and interrelationships between these components is what is
important.  Therefore we can abstract away the visual layout of a
spreadsheet and represent its structure as a dependence graph where
the vertices will be cells, charts, and other spreadsheet elements,
and the edges will represent the directed flow of information between
the nodes.  [For instance, if the cell B1 contains the formula
 \verb|"f(x) = A1 + A2"|, the dependence graph for that spreadsheet would
  look as shown in Figure 1.]

[Figure 1]

Constructing a dependence graph requires that we extract all
dependencies between the elements of a spreadsheet, and in order to do
that we have to examine every component and determine if it relies
on any other piece of data in the spreadsheet.  Dependencies are most
commonly manifested in formulas, in which case the cell containing the
formula is dependent on the cells that are arguments to the formula.
Dependencies can also be extracted from charts, where we can see that
a chart in the spreadsheet is dependent on all of its inputs.  Therefore
building a spreadsheet's dependence graph requires that we go through
all cells which contain formulas, and parse the formulas looking for
references to cells or ranges.

The standard convention for cell references in Excel is to use A1-style
coordinates, where 'A' designates the cell's column, and '1' designates
the row.  Columns are named in alphabetical fashion from A through Z,
after which follow columns AA, AB, AC, and so forth until AZ. Then come
columns BA, BB, BC, and so on.  The row coordinates are simply positive
integers.  There is also an alternative way for specifying cell
coordinates, called row-column notation, which has the form
R[integer i]C[integer j] -- simply referring to the cell in the i-th row
and j-th column.  Internally Excel uses the latter form, and so does our
parser.  Range references have the form [cell reference]:[cell reference].
One additional intricacy in referencing cells and ranges is the use of
absolute versus relative coordinates.  Preceding a coordinate by the
\verb|'$'| symbol designates it as absolute.  This means that if the formula
were to be copied and pasted into another cell, that coordinate would not
be changed, whereas otherwise Excel would automatically recompute it
relative to the new location of the formula.  Therefore cell references
may come in four different forms: \verb|A1, $A1, A$1, and $A$1|.  The same is
true for the endpoints of range references.

Additionally, Excel allows users to reference cells and ranges in other
worksheets of the same file. This is done with the syntax
\verb|[worksheet name]![cell or range reference]|. Since worksheet names may contain
empty spaces and a collection of special characters, worksheet names may
need to be surrounded by single quotes.  For example, a cell may contain the
formula \verb|"='Worksheet 1'!\$A\1".|

Extracting cell and range references from a formula is accomplished by
using regular expressions for recognizing the different forms of references
that may appear.  This is done starting with the most specific types of
references, removing any recognized string sequences from the formula,
and moving to the more general ones.  For instance, recognizing a range
reference is a more specific task than recognizing an individual cell
reference, since cell references are components of range references.  Consider
the formula \verb|"=SUM(A1:A5) + B1"|. If cell reference recognition were applied
before range recognition, we would extract the cells A1, A5, and B1, but instead
we should have recognized the range \verb|A1:A5|, and the cell \verb|B1|.

Cell and range reference recognition is performed by using the following
regular expressions in this order:
\begin{verbatim}
\"(\'\" + worksheet_name + \@"'!\$?[A-Z]+\$?[1-9]\d*:\$?[A-Z]+\$?[1-9]\d*)"  //Captures all references in the formula to ranges in the particular worksheet (worksheet_name); each item is a range reference of the form 'worksheet_name'!A1:A10 (tolerant of \$ signs)=
\"(" + worksheet_name + @"!\$?[A-Z]+\$?[1-9]\d*:\$?[A-Z]+\$?[1-9]\d*)"  //Captures all references in the formula to ranges in the particular worksheet (worksheet_name); each item is a range reference of the form worksheet_name!A1:A10 (tolerant of \$ signs)
\"('" + worksheet_name + @"'!\$?[A-Z]+\$?[1-9]\d*)" //Captures all references in the formula to cells in the specific worksheet (worksheet_name), where the reference has the form 'worksheet_name'!A1 (tolerant of \$ signs)
\"(" + worksheet_name + @"!\$?[A-Z]+\$?[1-9]\d*)" //Captures all references in the formula to cells in the specific worksheet (worksheet_name), where the reference has the form worksheet_name!A1 (tolerant of \$ signs)
\"(\$?[A-Z]+\$?[1-9]\d*:\$?[A-Z]+\$?[1-9]\d*)"  //Captures range references in formulas such as A1:A10 (tolerant of \$ signs)
\"(\$?[A-Z]+\$?[1-9]\d*)" //Captures single cell references such as A1 or \$A\$1, etc.
\end{verbatim}

The same analysis can be performed to extract cell and range references 
from the inputs to charts. In Excel, chart inputs are stored as Series objects, 
which can be accessed via the chart's SeriesCollection. The contents of a Series
are stored in its formula field as text, so the same type of analysis is applied
as with formulas in cells in order to extract references to cells and ranges of cells.

\subsection{Perturbation Analysis / Sensitivity Analysis / Cross-Validation}

Constructing the dependence graph is useful for two main
reasons. First, the process of extracting dependencies can provide
very useful information about the datasets in the spreadsheet without
needing complex layout analysis.  That is because whenever we identify
a reference to a range of cells in a chart's inputs or a formula,
we can generally assume that the cells in that range belong to the same dataset.  In
other words the creator of the spreadsheet has already identified this
range as containing contiguous data, and its cells are treated with no
distinction by the chart or formula.  As mentioned earlier,
identifying these sets of contiguous data is of great importance
because it provides the necessary context for further analysis.

The second significant benefit of the dependence graph is that it allows
us to pinpoint the spreadsheet elements that are most likely to be
important to the user.  These are the end results of the underlying
computations.  By looking for nodes in the graph that do not feed into
any other nodes we can identify the components of the spreadsheet most
likely to be the final outputs desired by the user.

Having knowledge knowledge of the datasets in a spreadsheet enables
the use of traditional statistical methods for outlier detection.
However, these methods typically require that the data's
distribution is known, which is generally not the case.  There are
also some statistical methods which are independent of the
distribution, but they are less sensitive.

Instead of employing statistical methods, we put to use our additional
knowledge of the outputs in the spreadsheet.  We are able to perform
perturbation (sensitivity) analysis via leave-one-out cross-validation
and measure the influence of cells on the final outputs.  The
cross-validation process works by sequentially replacing a value in
the range with every other value in the range and computing the
cumulative effect on the output.  This allows us to calculate the
importance, or influence, of that particular value on the value of the
output.  The effect on the output is calculated by summing the
absolute difference from the original after each substitution.

\[
\left| \frac{output_{old} - output_{new}}{output_{old}} \right|
\]

An important benefit of this type of analysis is that we do not have
to understand the underlying logic in the spreadsheet.  Instead we
simply perturb the inputs and look at the outputs.  This makes it
possible to analyze spreadsheets containing user-defined macros and
functions.

The perturbation analysis is less clear when dealing with outputs
which are not numeric.  In the case of text outputs in cells, the
change is treated as a boolean.  That is, if perturbing a value causes
a change in the text output, we add 1 to the total influence;
otherwise it does not change.  In the case of visual outputs, such as
charts, it is more difficult to measure the effect.  For certain kinds
of charts, such as bar graphs or pie charts, we are inclined to expect
to see equally distributed data (bars and wedges of approximately the
same size). Therefore in those cases we measure the perturbation
effect by the amount the result deviates from equilibrium.

\section{Limitations}

Our approach relies on the presence of formulas or charts to establish
some relationships between the data.  However, we observe that in the
spreadsheet corpus, which contains a collection of near 5000
spreadsheets from various sources, about 40\% of them contain
formulas, giving reason to believe that formulas are commonplace and
we can expect to be able to rely on them~\cite{Fisher:2005:ESC:1082983.1083242}.
}
