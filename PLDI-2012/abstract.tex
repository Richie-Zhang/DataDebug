Testing and static analysis tools can help root out bugs in programs,
but not bugs in data. Checking data for errors is arguably as
important as finding program errors, but lacks effective tool
support. This paper introduces \emph{data debugging}, an approach that
combines dataflow dependence analysis with statistical analysis to
locate likely data errors. Since it is impossible to know \emph{a
priori} whether data are erroneous or not, data debugging does the
next best thing: locating data where an error would have the most
impact. Data debugging is particularly promising in the context of
data-intensive programming environments like databases and
spreadsheets, which intertwine data with programs (in the form of
queries or formulas).

This paper presents an implementation of data debugging in an add-in
tool for Microsoft Excel called \checkcell{}. \checkcell{} highlights
unusually impactful values in shades of red proportional to their
effect on the spreadsheet's computation (including charts and 
formulas). \checkcell{} is efficient: its algorithms are asymptotically
optimal, and it takes under a minute to run on large spreadsheets. A
user study verifies the effectiveness of data debugging to find
data errors. \checkcell{} users were able to find errors with XX\%
accuracy, while users without were only able to achieve YY\% accuracy.
