Testing and static analysis tools can help root out bugs in programs,
but not bugs in data. Checking data for errors is arguably as
important as finding program errors, but there currently exists no
effective tool support. Currently, the only approach is manual
inspection of data, one at a time. Because inspection is onerous and
ineffective at scale, most data remains full of errors. CITE
SPREADSHEET STUDY OF ERROR RATES.

This paper introduces ``data debugging'', which combines program
analysis with data analysis. Data debugging aims to locate values that
have an unusual effect on the final computation.  These values not
only provide important insights into the data but also can reveal data
errors.  Data debugging is particularly promising in the context of
data-intensive programming environments where data and programs are
intermingled, like databases (queries and stored procedures) and
spreadsheets (formulas).

We present the first data debugging tool, CheckCell, which operates on
spreadsheets. CheckCell builds a dependency graph of an entire
spreadsheet, including formulas and charts, where the leaves are cells
or ranges of cells. It then computes the influence of every cell by
systematically evaluating the impact of replacing it with any other
item from the same range. Because data errors only matter when they
have a significant impact, CheckCell highlights important values in
shades of red proportional to their influence in the spreadsheet.  We
perform a user study via to measure the effectiveness of using
CheckCell to find injected errors in spreadsheets. CheckCell users
were able to find errors with XX\% accuracy, while users without
CheckCell were only able to achieve YY\% accuracy.
