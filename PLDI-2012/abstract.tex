Testing and static analysis tools can help root out bugs in programs,
but not bugs in data. Checking data for errors is arguably as
important as finding program errors, but lacks effective tool
support. Previous approaches like data cleaning and statistical
outlier analysis require either ground truth data for
cross-validation, or that the data follow a known statistical
distribution.

This paper introduces \emph{data debugging}, an approach that combines
data dependence analysis with statistical analysis to find and rank
likely data errors. Since it is impossible to know \emph{a priori}
whether data are erroneous or not, data debugging instead
reveals data whose impact on the computation is unusual.
Data debugging is particularly promising in the context of
data-intensive programming environments that intertwine data with
programs (in the form of queries or formulas).

This paper presents the first data debugging tool, \checkcell{}, an
add-in for Microsoft Excel. \checkcell{} highlights values in shades
proportional to the unusualness of their impact on the spreadsheet's
computation, which includes charts and formulas. \checkcell{} is efficient;
its algorithms are asymptotically optimal, and the current prototype
runs in seconds for most spreadsheets we examine. We perform a case
study by employing workers via a crowdsourcing platform, and show
that \checkcell{} is effective at finding actual data entry errors.

%A user study verifies the effectiveness of data debugging to
%find data errors. Users without \checkcell{} were only able to find
%errors with XX\% accuracy, while users with \checkcell{} were able to
%achieve YY\% accuracy.
