Testing and static analysis tools can help root out bugs in programs,
but not bugs in data. Checking data for errors is arguably as
important as finding program errors, but lacks effective tool
support. Previous approaches like data cleaning and statistical
outlier analysis require either ground truth data for
cross-validation, or that the data follow a known statistical
distribution.

This paper introduces \emph{data debugging}, an approach that combines
dataflow dependence analysis with statistical analysis to find and
rank likely data errors. Since it is impossible to know \emph{a
priori} whether data are erroneous or not, data debugging does the
next best thing: locating data where errors would have the most
impact. Data debugging works by building a dependence graph, and then
measures impact by observing the effect of replacing items with
randomly chosen data from the same group, such as columns in databases
or ranges in spreadsheets. This approach lets data debugging find
errors in both numeric and non-numeric data.

Data debugging is particularly promising in the context of
data-intensive programming environments like databases and
spreadsheets, which intertwine data with programs (in the form of
queries or formulas). This paper presents the first data debugging
tool, \checkcell{}, an add-in for Microsoft Excel. \checkcell{}
highlights unusually impactful values in shades proportional to their
effect on the spreadsheet's computation, including charts and
formulas. \checkcell{} is efficient; its algorithms are asymptotically
optimal, and the current prototype runs in seconds. We show
that \checkcell{} is able to find injected errors in a suite of real
spreadsheets.

%A user study verifies the effectiveness of data debugging to
%find data errors. Users without \checkcell{} were only able to find
%errors with XX\% accuracy, while users with \checkcell{} were able to
%achieve YY\% accuracy.
