\thispagestyle{empty}
\singlespace

\begin{center}
        {\large\bf\textsf{
            University of Massachusetts \\ Budget Justification}}
\end{center}

\subsection*{Senior Personnel}
\begin{itemize}
\item PI Emery Berger:  1 summer month/yr.
\item Co-PI Alexandra Meliou: 0.75 summer months/yr.
\end{itemize}

A 3.5\% increase is added per year.

\subsection*{Other Personnel}
\begin{itemize}
\item 2 Graduate Students: Student salary is based on approximately \$27.34 an hour, 
 20 hours a week, 38 weeks per academic year.
 A 3.5\% increase is added per year. 
\end{itemize}

\subsection*{Fringe Benefits}

\begin{table}[!h]
\centering
\begin{tabular}{|l|l|}
\hline
\emph{Benefited Positions / Fringe Rates:} &  \\
Fringe 7/1/12--6/30/13 &   25.98\% + workers comp. 0.53\%, +      \\
                       &   UI, UHI, MTX 1.29\% = \textbf{27.80\%} \\
\hline
Health \& Welfare      &   \$14.50 weekly = \$754 annually        \\
\hline
Sick Leave Bank        & 0.30\%, not assessed on Faculty Salaries \\
\hline
\end{tabular}
\end{table}

\begin{itemize}

\item \textbf{Fringe} benefits applicable to direct salaries and wages
  are treated as direct costs. They are the rates identified in the
  Massachusetts Statewide Cost Allocation Plan approved by DHHS. This
  rate is comprised of Group Insurance and Retirement. The combined
  rate must be applied to all benefited personnel on any awards.

\item \textbf{Health and Welfare} (H \& W) for all benefited positions
  is \$14.50 per week (\$754) annual FTE (prorate on part-time
  positions).  Health and Welfare must be assessed to all graduate
  student appointments.  A 5\% increase is added per year.

%\item \textbf{Summer Student Payroll:} \emph{All students} (excluding
%  Post Docs and Fellows) employed for the summer and not enrolled in
%  classes are to be assessed 1.29\% UI, UHI, MTX on the summer salary.

\item \textbf{GEO Health Deferment Rate:} An appointment that is 20
  hours per week both summer and academic year (1,040 hrs) would be
  assessed \$5,439.20.  A 5\% increase is added per year.

\end{itemize}

\subsection*{Other Direct Costs}

\begin{itemize}
\item \textbf{Domestic Travel:} Anticipate travel to the following
  conferences: PLDI, ICSE, VLDB, etc. Exact dates and locations to
  be determined.  A 5\% increase is added per year.
 
  Years 1--3: One conference at approximately \$1,750 for each of the PIs and
graduate students. Approximate
costs include \$450 RT airfare, \$750 registration fee, \$125/night hotel (2 nights),
ground transportation \$200, per diem \$100.

%\item \textbf{International Travel} – Anticipate travel to the
%  following conferences: PLDI, POPL, ICSE, FSE (overseas). Exact dates
%  and locations to be determined.  A 5\% increase is added per year.

%  Years 1--3: 1 conference at approximately \$5,000 for one PI and one
%  graduate student. Approximate costs include \$2,500 RT
%  airfare, \$800 registration fee, \$800--\$1,000 foreign per diem
%  rates (4 days), \$350 ground transportation, \$350 miscellaneous.
 
\clearpage
\thispagestyle{empty}

% \item \textbf{Materials and Supplies:}

%  One laptop at approximately \$1,200 will be purchased in each year
%  for a graduate student.  The laptop is project-specific because of
%  the need to experiment on state-of-the-art browsers and computing
%  environments. Purchase of computer peripherals, upgrades, storage
%  and parts for the laptop.  A 5\% increase is added per year.

\item \textbf{Computer Costs:}

  Computer maintenance will be paid to the Computer Science Department's Computer Facility to provide maintenance and support of equipment, software, and communication networks, as well as, mass storage and file back-up services. The charges are based on a campus approved fee structure.  There are also general computer science facility maintenance charges that are billed according to FTE faculty, staff, and students assigned to the project (for central/shared services) and according to the number of workstations and other general-purpose equipment assigned to the project (for maintenance of this equipment).  A 5\% increase is added per year.

\item \textbf{Other:}

  The University charges a curriculum fee to all grad student
  appointments.  FY’14 -- \$8,618.40. 
% Summer appointments are not   charged curriculum fee.
  This fee is exempt from I.C.  A 5\%
  increase is added per year.

\end{itemize}

\subsection*{Indirect Costs}

Indirect costs are 59\% MTDC, 7/1/12--6/30/15.


\clearpage
