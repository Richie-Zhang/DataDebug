\section{Analysis}
\label{sec:analysis}

% anomaly issue: we want to compare the effect of making 30 (K) random choices to 
% systematic exploration of the space (as in, Sum_{j != i}{|effect(a_j)-effect(a_i)} / N).
% compare false positives (we have non-anomaly but by bad luck, we exchange with anomalous impacts),
% and false negatives (we have an anomaly but by bad luck, we exchange with anomalous impacts).

This section presents an analysis of several aspects of data
debugging: its runtime, its sampling approach to measure impact for
large data ranges, and its scoring to perform impact outlier analysis.

\subsection{Asymptotic Runtime}
\label{sec:asymptotic_analysis}

Data debugging operates in several phases: computing the dependence
graph, performing impact analysis, and then ranking impacts. Runtime
depends on the following parameters: the number of data items ($n$),
the number of formulas ($f$), and the number of inputs ($i$). In
spreadsheets, the number of inputs equals the number of data items,
but in other contexts like databases, inputs correspond to fields, so
$i \ll n$. Since each formula and data item must be examined at least
once to compute the dependence graph and to measure impact,
respectively, runtime for data debugging must be 
$\Omega(n+f)$.

The cost of building the dependence graph varies depending on the
structure of the computation. It has a worst-case runtime of
$O((i*f)^2)$, quadratic in the total number of inputs and formulas; it
is theoretically possible for each formula to depend on every input
and other formula. However, this kind of pathological computation
structure is atypical. Dependence graphs normally form a tree or a
forest of trees. In this case, the cost of constructing the dependence
graph becomes linear in the number of inputs and formulas, or
$O(i+f)$.

Impact analysis dominates the costs of constructing the dependence
graph and ranking impacts, since it requires recalculation of the
computations affected by changes in the data.

A na\"ive implementation of impact analysis that checked the impact of
each data item by systematically replacing it with every other item in
the same range would require $O(n^2)$ time. Worse, each of these
iterations requires potentially costly recalculations. For large
ranges, such an approach would make data debugging unusable in
practice.
 
By using a fixed number of random selections once the range
exceeds a threshold size, data debugging keeps the total number of
recalculations to $O(n)$, linear in the number of data items. Any
strategy that visits each data item at least once takes $O(n)$ time,
so this bound is tight. Section~\ref{sec:sampling_effectiveness}
explains why this strategy approximates the effect of a complete
examination of other inputs, while minimizing execution time.

Finally, ranking impacts involves only two linear passes over the
impacts to compute absolute impact scores, so it also operates in
(optimal) linear time in the number of impacts.

\subsection{Sampling Effectiveness}
\label{sec:sampling_effectiveness}

A complete strategy for measuring the impact of any data item $i$ is to
systematically measure the impact of replacing it with every other
item $j$ in the same range:

\begin{equation}
\mbox{impact}_i = \frac{\sum_{j \neq i}^{N}{|\mbox{result}_i-\mbox{result}_j|}}{N}
\end{equation}

\noindent
This approach, as mentioned above, would take $O(n^2)$ time and so is
inefficient for large data sets, though it is reasonable to do for
small ones.

For large data sets, data debugging employs a sampling-based strategy
that randomly chooses (with replacement) a fixed number of samples $K$
(e.g., $K = 30$).

\begin{equation}
\mbox{estimated\_impact}_i = \frac{\sum_{j \in \{\mbox{sample}\}}^{N}{|\mbox{result}_i-\mbox{result}_j|}}{K}
\end{equation}

The following theorem establishes that using the estimated impact is
likely to result in minimal error.

\begin{theorem}
When the percentage of values with unusual impacts is low, the estimated impact closely approximates the actual impact because it minimizes the risk of false positives and false negatives.
\end{theorem}

\noindent
There are two cases where using the estimated impact will have a different effect than using the actual impact:

\begin{itemize}
\item \textbf{False negatives.} The item under consideration has an
  unusual actual impact, but the sampling procedure repeatedly
  chooses other items with similarly anomalous impacts, so the item under
  consideration appears to have only average impact (the anomalous values cancel out).
\item \textbf{False positives.} The item under consideration does
  not have an unusual actual impact, but the sampling procedure
  repeatedly chooses items that do have an unusual impact. Now, the item
  under consideration appears to have an anomalous impact, despite the fact that it actually does not.
\end{itemize}

The key to bounding the likelihood of either false negatives or false
positives is to ensure that the sampling process does not repeatedly
sample data items with unusual impacts.

Recall that data debuggging considers a data item to have an unusual
impact when that impact is at least two standard deviations away from
the mean impact in that range. When impacts follow a normal
distribution, the number of items with unusual impact will be less
than 5\%. Of course, impacts will not necessarily be normally
distributed, although they will be when the computations include
averages (by virtue of the Central Limit Theorem). Nonetheless, as
long as the tails of the impact distribution comprise a small fraction
of the total, the theorem holds, as we explain below.

One can view the sampling procedure as a series of Bernoulli
trials, repeatedly flipping a biased coin with a probability $p =
0.05$ of choosing a value whose impact is anomalous (heads) and a
probability $q = 0.95$ of choosing a value with a non-anomalous impact
(tails). For $n$ coin flips and probability $p$ of heads, the expected
number of times that the sampling procedure will choose an anomalous
value is just $np$, or 1.5.

Because the Poisson distribution is an excellent approximation as long
as $n$ is at least 20 and $p$ is no more than $0.05$, it is convenient
to use it to approximate the likelihood of choosing a larger number of
high-impact values than $x$, where $x > \lambda$ and $\lambda = np$:

\begin{equation}
Pr[X \geq x] \leq \frac{e^{-\lambda}(e \lambda)^x}{x^x}
\end{equation}

We can use this equation to show that it is highly unlikely that the
sampling procedure would accidentally choose a large number of
unusually-impactful values. For example, the probability of choosing
one-third of those values from a sample of size $K = 30$ is less than
$3/100,000$.

Therefore, the estimated impact computed by sampling is
nearly as effective at reducing both false positives and false
negatives as computing the actual impact.

\subsection{Impact Outlier Analysis}
\label{sec:outlier_analysis}

Once all impact scores have been computed, only those data whose
impacts cross some threshold of anomalousness should be
reported. We currently treat the impact scores as if they fit a normal
distribution, and only report data whose scores place them more
than two standard deviations away from the mean. In a normal
distribution, that is just under 0.05\% of the population; in other
words, this corresponds to a 95\% confidence level that these are
anomalies.

In the absence of knowledge of the distribution of impact scores, using
the normal distribution is a conservative approach. That is, it
is likely to report few false negatives, at the risk of introducing some false
positives. This fact derives from two particular characteristics of the
normal distribution: its low \emph{skewness} and \emph{kurtosis}.

The normal distribution has zero skewness, where skew is the
distribution around the mean; in other words, it is perfectly
symmetric. Any asymmetric distribution by definition has a greater
number of points either to the left or to the right of the mean. By
choosing outliers from the tails of the normal, \checkcell{} also
includes one of the skewed tails of any asymmetric distribution.

In addition, the normal has either zero kurtosis, which indicates the
peakedness of the curve and heaviness of the tail. Counting outliers
from the perspective of the normal distribution is generally
conservative, since it includes all distributions with heavier tails
(i.e., those with positive kurtosis). Because distributions with negative
kurtosis generally have small tails, they also tend to have few
outliers, by definition. Thus, failing to report outliers for
distributions with negative kurtosis is usually safe.
One exception worth noting and a limitation of this approach is
symmetric multimodal distributions, where the outlier values are not only
in the tail but also centered around the mean. Using the normal as a
reference would not uncover such outliers.

% we just need to report outliers in the impact. assuming the impacts
% are normal is a conservative approach: the normal has 0 skewness
% (skew = distribution around the mean -- normal is symmetric, so 0
% skew) and low (either 0 or 3) kurtosis, depending on your definition
% of kurtosis. Every non-normal distribution is by definition more skewed and most
% distributions have a higher kurtosis (heavier tails), so we may
% overmark outliers. We won't find outliers in distributions with
% negative kurtosis, but those are super weird (no tails -- they drop
% below the x-axis at some point on either side), so it's hard to
% argue that they have outliers at all.
