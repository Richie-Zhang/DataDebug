\section{Medium Scale Data Debugging: Relational Databases} % (fold)
\label{sec:bi_directional}

We propose to apply data debugging techniques to perform data cleaning
on relational data.  Cleaning tools attempt to purge datasets of
discrepancies before the data can be used, but many errors still go
undetected and get propagated further through queries and other
transformations. When errors are detected in transformation results,
it is critical to trace them back to their source and correct them to
prevent them from affecting other computations. Data debugging can
facilitate \emph{post factum} cleaning by identifying the input data
with the highest contributions to an incorrect result. We propose to
combine two methods of impact analysis in large-scale relational
datasets: forward and reverse analysis.

\subsection{Forward Analysis} % (fold)
\label{sub:forward_analysis}

The approach outlined in Section~\ref{sec:overview} relies on
re-calculating computations either exhaustively, or using random
sampling over a range of data. In a database setting, this approach
can be viewed as an extreme form of \emph{forward analysis}: an
analysis that starts with data and observes its effect on a
query. Previous forward analyses in database research include
\emph{hypothetical}, or \emph{what-if} queries
\cite{DBLP:conf/vldb/BalminPP00,DBLP:conf/icde/LakshmananRS08}: a
query result is computed based on a (single) hypothetical change to
the input data. Data debugging requires invoking multiple what-if
queries to find values with inordinately high impacts.

Forward analysis is powerful because it does not require any knowledge
of the inner workings of the computation, but for database systems,
directly adopting the data debugging approach outlined earlier would
be inefficient. Computing the transformation itself may be
computationally intensive, and calculating the exact impact of each
input tuple would require an exhaustive search over the input domain.

One way to reduce the cost of data debugging in databases is to peel
back the black box abstraction, as described in our proposed work for
spreadsheets.  The black box abstraction makes it particularly
suitable for user-defined functions (UDF). However, many computations
in the relational model rely on SQL transformations, which are simple
to analyze.

% Exploiting knowledge of the data transformations can
%improve performance, but requires lifting the black box assumption.

% In this setting, the black box model is too restrictive.


% Forward analysis is inefficient and does not scale to complex problems: computing the transformation itself may be computationally intensive, and calculating the exact impact of each input tuple requires an exhaustive search over the input domain. 
% Calculating impact over sets of input tuples makes the problem even harder, as it causes combinatorial explosion of the search space. 
%Lifting the black box assumption can often do much better. 
% We revisit the personal budget example discussed in Section~\ref{sec:overview}. If we know the function that computes cell \texttt{B12}, we can calculate the impact of a cell \texttt{B$_j$} as follows:
% \begin{align}\label{eqn:example_reverse}
% 	\sum_{i\neq j} I(\texttt{B}_j - \texttt{B}_i > \texttt{B}11-\texttt{C}11-150) = \sum_{i\neq j} I(\texttt{B}_j - \texttt{B}_i >3520.92)
% \end{align}
% where $I$ is an indicator function. Therefore, we can determine the impact of an input value without recomputing the function. SQL transformations are particularly amenable to such optimizations. 

Consider the following example SQL query, where the inner subquery
returns all departments that have a branch in Boston, and the outer
query computes the average salary of the employees that do not work in
any of those departments.

%The following nested SQL query returns the average salary of all employees who do not work for a department that has a branch in Boston.
\texttt{
\begin{tabbing}
		SELECT	\=avg(E.salary)\\
		FROM	\>Employees E\\
		WHERE	\>E.department NOT IN	(\=SELECT	\=distinct D.name\+\+\\
										FROM	\>Department D, Branches B\\
										WHERE	\>D.did = B.did AND B.location = `Boston')
\end{tabbing}}

\noindent
In this example, we do not need to recompute the nested subquery if
only the \texttt{Employees} table is modified during impact
analysis. SQL queries are particularly amenable to such optimizations.

% \comment{In a relational setting, impact analysis also needs to account for functional dependencies and other database-enforced constraints (e.g. foreign-key constraints). These would restrict the modifications that we can apply to the input. Is this interesting/important enough to mention here?}

% \comment{Possible argument against black box: ``weird'' functions like parity.}

% subsection forward_analysis (end)

\subsection{Reverse Analysis} % (fold)
\label{sub:reverse_analysis}
Forward impact analysis is not well-suited for certain errors that
result from data integration. To address these cases, we propose \emph{reverse analysis}. This technique analyzes transformations, such as SQL queries, and effectively inverts them to derive an input from a given output.

Data in databases often comes from multiple
sources, and is collected via multiple methods. In practice, databases are
often produced by merging other existing databases, and this process can
introduce redundancies and inconsistencies. Data integration errors are often identified by the violation of key constraints and other functional dependencies. Figure~\ref{fig:integrationExample} demonstrates a key-constraint violation after merging two relations. A database needs to satisfy several, often complex constraints, and the cleaning task may specify additional constraints or objectives.

\begin{figure}
	\small{\begin{tabular}{|l|l|l|l|l|l|l|l|}
		\multicolumn{5}{c}{Input tables} & \multicolumn{1}{l}{}& \multicolumn{2}{c}{Integration result}\\
		\multicolumn{5}{c}{$\overbrace{\rule{9.5cm}{0pt}}$} & \multicolumn{1}{l}{}& \multicolumn{2}{c}{$\overbrace{\rule{4.5cm}{0pt}}$}\\
		\multicolumn{2}{l}{Products (A)} & \multicolumn{1}{l}{} & \multicolumn{2}{l}{Products (B)} & \multicolumn{1}{l}{} & \multicolumn{2}{l}{All Products}\\
		\cline{1-2}\cline{4-5}\cline{7-8}
		\textbf{\underline{ItemNum}} & \textbf{product name} & \multicolumn{1}{l|}{} & \textbf{\underline{ItemNum}} & \textbf{product name} & \multicolumn{1}{l|}{} & \textbf{ItemNum} & \textbf{product name}\\
		\cline{1-2}\cline{4-5}\cline{7-8}
		5253B002 & Nikon D4 SLR & \multicolumn{1}{l|}{} & 5253B002 & Nikon D4 digital & \multicolumn{1}{l|}{} & 5253B002 & Nikon D4 SLR\\
		\cline{1-2}\cline{4-5}\cline{7-8}
		25482 & Canon EOS 1D X & \multicolumn{1}{l|}{} & 25482 & Canon 1DX EOS & \multicolumn{1}{l|}{} & 5253B002 & Nikon D4 digital\\
		\cline{1-2}\cline{4-5}\cline{7-8}
		\multicolumn{6}{l|}{} & 25482 & Canon EOS 1D X\\
		\cline{7-8}
		\multicolumn{6}{l|}{} & 25482 & Canon 1DX EOS\\
		\cline{7-8}
		
	\end{tabular}}
	\caption{ItemNum is a key (its value is unique) in the input tables, but the same is not true for the integration result. The integrated data needs to be cleaned to produce a valid data instance.}\label{fig:integrationExample}
\end{figure}

\begin{example}\label{ex:how-to}
A manufacturing company orders parts from multiple suppliers around
the world. To reduce its dependence from any single country, the
company makes sure that its inventory orders from each country do not
exceed 10\% of all orders. However, after merging with another
manufacturer, this condition is no longer satisfied. The company needs
to ``clean'' its inventory orders by reassigning some orders to
different suppliers. Ideally, it would like to achieve this with the
minimum number of changes.
\end{example}

Example~\ref{ex:how-to} presents a data cleaning scenario which is not
well suited for forward analysis. In the presence of complex
conditions, forward impact analysis has to exhaustively search all
possible scenarios.

In previous work, we have modeled these scenarios with \emph{how-to}
queries: ``how should the input change in order to achieve the desired
output?''~\cite{DBLP:journals/pvldb/MeliouGS11} How-to queries are an
example of reverse analysis. In contrast to forward analysis,
reverse analysis specifies the desired effect on the output, and
reverse engineers how the input should change. Reverse analysis has
more complex semantics and higher implementation complexity compared
to forward analysis due to the fact the inverse of a function is not
always a function. Given a desired output, there may be multiple input
values (or none at all) that produce it.  As opposed to forward
analysis, reverse analysis cannot handle black box computations, but
uses knowledge of the transformations to perform the evaluation
efficiently.

%To circumvent this difficulty in a general framework we will use SAT and Mixed Integer Programming (MIP) solvers as general purpose tools in reverse analysis. 

% We propose to augment data debugging with \emph{reverse impact analysis}. In contrast to forward analysis (``how would the output change for a given change in the input''), reverse analysis answers the question ``how should the input change in order to achieve the desired output?'' 

The Tiresias system~\cite{DBLP:conf/sigmod/MeliouS12,
  DBLP:conf/sigmod/MeliouSS12}, developed by PI Meliou, can
efficiently evaluate how-to queries, and is used for complex cleaning
tasks such as the one in Example~\ref{ex:how-to}. How-to queries are
constraint optimization problems defined over relational data, but
they cannot be expressed in standard SQL. Tiresias allows users to
issue how-to queries through a declarative language based on
Datalog~\cite{DBLP:journals/tkde/CeriGT89}, and the queries are
evaluated using mixed integer programming solvers.

\paragraph{Calculating impact.} % (fold)
\label{par:calculating_impact}

Tiresias does not quantify the impact of input data, but rather
produces an instance of the input that satisfies all cleaning
constraints. In prior work, PI Meliou introduced a theory of causality
for relational databases~\cite{DBLP:journals/debu/MeliouGHKMS10},
which can model impact under the reverse analysis model. It is
possible to characterize the family of conjunctive queries for which
\emph{responsibility}---the contribution of an input to a result---can
be computed in polynomial
time~\cite{DBLP:journals/pvldb/MeliouGMS11}. Tiresias uses SAT solvers
to compute responsibility efficiently for NP-hard
cases~\cite{DBLP:conf/sigmod/MeliouGNS11}.

% paragraph calculating_impact (end)

% Halpern and
%Pearl~\cite{HalpernPearl:Cause2005} and Chockler and
%Halpern~\cite{DBLP:journals/jair/ChocklerH04} gave mathematical
%definitions of causality and its related notion of degree of
%responsibility. Responsibility quantifies  

% subsection reverse_analysis (end)


%--------------------------------------------------------------------------

\subsection{Proposed Work} % (fold)
\label{sub:research_plan}

Complex transformations contain both glass box and black box
components. We propose to develop a \textbf{bi-directional data
  cleaning} framework that combines both forward and reverse
analysis. This framework can flexibly adapt to different types of
computations. We will investigate the following research questions:

\paragraph{Static analysis of SQL queries.} % (fold)
\label{par:static_analysis_of_sql_queries}
SQL queries are the primary type of computation in relational databases. We will investigate techniques to break down queries into \emph{active} and \emph{inactive} components: active components are the parts of the query that need to be recomputed during forward analysis, whereas inactive are components that compute intermediate results that don't change and thus can be precomputed. This is a hard problem as it relates to the problem of query containment, which is undecidable~\cite{DBLP:conf/pods/CalvaneseGL98}.
% paragraph static_analysis_of_sql_queries (end)

\paragraph{Cleaning optimizations.} % (fold)
\label{par:cleaning_optimizations}
In complex transformations, it is not clear how forward and reverse analysis
should be combined. We will use computational patterns to design a cleaning
optimizer, modeled after traditional relational query optimizers. Given a
cleaning task, the optimizer should select how to subdivide the task into the
forward and reverse components, and in which order the tasks should be
performed.
% paragraph cleaning_optimizations (end)

\paragraph{Interactive cleaning.} % (fold)
\label{par:interactive_cleaning}
Cleaning tasks are often based on complex constraints, and have many possible
solutions. It can be a daunting undertaking for a user to assess these
solutions, or even define the cleaning task itself. We will design an
interactive interface to allow users to easily identify possible problems, and
we will investigate techniques to refine solutions based on user feedback.
% paragraph interactive_cleaning (end)



% subsection research_plan (end)



% section bi_directional (end)
