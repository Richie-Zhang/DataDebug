\paragraph{Emery D. Berger} is a PI of three active NSF grants.

% Thus far, 8 graduate students have been funded in
% full or in part by these grants; one graduated with a PhD in 2006, and
% one in 2009.

% EAGER

Professor Berger is a PI in a collaborative project with Professor
Daniel Jim\'enez (Texas A\&M University): {\em SHF: Large:
Collaborative Research: Reliable Performance for Modern Systems} (NSF
CCF-1012195, \$550,000, August 2010-July 2013). This project aims to
deliver reliable performance on modern computer systems. By
introducing randomness into the way a computer runs programs, a
reliably performant system will significantly reduce the probability
that any small change will have a large impact on performance.  It has
thus far produced two publications by PI Berger, to appear at ASPLOS
and in ACM TECS ~\cite{proartis2013,stabilizer:asplos13}, and one
major software release: \textsc{Stabilizer}, a compiler and runtime
system that enables statistically sound performance evaluation.

Professor Berger is sole PI of {\em Programming the Crowd} (NSF
CCF-1144520, \$300,002, August 2012-December 2013). This project
introduces crowdprogramming, an approach that fully integrates human
and digital computation. In crowdprogramming, humans are modeled as
function calls in a standard programming language. This approach lets
programmers focus on programming logic, while the crowdprogramming
runtime system manages the critical tradeoffs between cost, time, and
data quality. We have released a crowdprogramming system
called \textsc{AutoMan} and reported on it at
OOPSLA~\cite{Barowy:2012:API:2384616.2384663}.

Professor Berger is a PI in a collaborative project with Professor
Michael Hicks (University of Maryland) and Professor Kathryn McKinley
(UT-Austin): {\em PASS: Perpetually Available Software Systems} (NSF
CCF-0910883, \$639,420, August 2009-July 2013). This project proposes
a transformative paradigm shift to ``perpetually available software
systems'' (PASS) that will make software more available and robust by
directly addressing errors in deployed software. Thus far, this
project has led to two publications by PI Berger at OOPSLA and
SOSP~\cite{Liu:2011:DED:2043556.2043587,Liu:2011:SPD:2048066.2048070},
and two major software releases: \textsc{Dthreads}, a deterministic
replacement for \texttt{pthreads}, and \textsc{Sheriff}, a system that finds or eliminates false sharing.

PI Berger has been supported by three previous NSF grants:

The first of these grants is
\emph{CAREER: Cooperative System Support for Robust High Performance}
(NSF CAREER CNS-0347339, \$477,000, June 2004-2009). The purpose of this project is to attack the growing
latency between main memory and disk with adaptive algorithms that
leverage cooperation between the compiler, runtime systems, and
operating system. It has thus far produced eight papers that have
appeared at PLDI, OSDI (2), OOPSLA, ISMM, MSP, and USENIX~\cite{feng05,hert05,hert05a,yang04,1267316,transparent2006usenix,flux06usenix,DBLP:conf/osdi/YangLBKM08}.

PI Berger was the sole PI of \emph{Probabilistically Correct
  Execution: Hardening Applications Against Error and Attack} (NSF
  CNS-0615211, \$300,000, September 2006-2009, no cost extension).
  Probabilistically correct execution is an approach that allows
  programs to execute correctly with high probability in the face of
  errors or attack, and functions by both \emph{randomizing} the
  runtime system and \emph{replicating} differently-randomized
  executables. This combination transforms unreliable software
  components into a quantifiably more reliable system. This grant has
  thus far produced four papers, two at PLDI, one at ASPLOS, and one
  at
  OOPSLA~\cite{1134000,1250736,1346296,DBLP:conf/oopsla/BergerYLN09}. One
  result of this work is DieHard, a freely-available system that
  increases the security and resilience to memory errors of
  C/C++-based applications; DieHard has been downloaded over 20,000
  times since its release, and directly inspired the Fault-Tolerant
  Heap included in Windows since version 7. Another product of this
  research, the DieHarder secure heap, an inspiration for the security
  hardening features incorporated into Windows 8.

PI Berger was also a co-PI of \emph{MRI: Cluster Acquisition for
Computational Research into Large Scale Data Rich Problems} (NSF
CNS-0619337, \$350,000, 9/1/2006--9/1/2008).

