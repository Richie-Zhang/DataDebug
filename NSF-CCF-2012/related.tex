\section{Related Work}

% The formula here is: discuss related work in one facet; end with a contrast with the current work.

\subsection*{Data Cleaning}

Most past work on locating or removing errors in data has focused
on \emph{data cleaning} or \emph{scrubbing} in database
systems~\cite{DBLP:journals/debu/RahmD00,han2006data}. Standard
approaches include statistical outlier analysis for removing noisy
data~\cite{1583581}, interpolation to fill in missing data (e.g., with
averages), and using cross-correlation with other data sources to correct or
locate errors~\cite{Hernandez:1995:MPL:223784.223807}.

% Also: noise removal.

% Section~\ref{FIXME} shows that statistical outlier analysis often produces unacceptably large numbers of false positives. 

% SURVEY! http://www.dbis.informatik.hu-berlin.de/dbisold/research/bioinformatics/papers/data_cleansing.html

A number of approaches have been developed that allow data cleaning to
be expressed programmatically or applied interactively. Programmatic
approaches include AJAX, which expresses a data cleaning program as a
DAG of transformations from input to
output~\cite{Galhardas:2000:AED:342009.336568}. Data Auditor applies
rules and target relations entered by a
programmer~\cite{Golab:2010:DAE:1920841.1921060}. A similar
domain-specific approach has been employed for data streams to smooth
data temporally and isolate it spatially~\cite{1617508}. Potter's
Wheel, by Raman and Hellerstein, is an interactive tool that lets
users visualize and apply data cleansing
transformations~\cite{Raman:2001:PWI:645927.672045}.

To identify errors, Luebbers et al. describe an interactive data
mining approach based on machine learning that builds decision trees
from databases. It derives logical rules (e.g., ``$\mbox{BRV} =
404 \Rightarrow \mbox{GBM} = 901$'') that hold for most of the
database, and marks deviations as errors to be examined by a data
quality engineer~\cite{Luebbers:2003:SDD:1315451.1315499}. Raz et al.\
describe an approach aimed at arbitrary software that uses
Daikon~\cite{ernst2007daikon} to infer invariants about numerical
input data and then report discrepancies as ``semantic
anomalies''~\cite{Raz:2002:SAD:581339.581378}.  Data debugging is
orthogonal to these approaches: rather than searching for latent
relationships in or across data, it measures the interaction of data with
the programs that operate on them.
 

\subsection*{Spreadsheet Errors}

Spreadsheets have been one of the most prominent computer applications
since their creation in 1979.
 The most widely used spreadsheet application today is Microsoft
Excel. Excel includes rudimentary error detection including errors in
formula entry like division by zero, a reference to a non-existient
formula or cell, invalid numerical arguments, or accidental mixing of
text and numbers.
% http://office.microsoft.com/en-us/excel-help/find-and-correct-errors-in-formulas-HP010066255.aspx
Excel also checks for inconsistency with adjacent formulas and other
structural errors, which it highlights with a ``squiggly'' underline. In addition, Excel provides a formula auditor, which lets users view dependencies flowing into and out of particular formulas.
% http://office.microsoft.com/en-us/excel-help/use-error-checking-to-correct-common-errors-in-formulas-HA010342331.aspx

Past work on detecting errors in spreadsheets has focused on inferring
units and relationships (has-a, is-a) from information like structural
clues and column
headers, and then checking for inconsistencies~\cite{Antoniu:2004:VUC:998675.999448,DBLP:conf/kbse/AhmadAGK03,Chambers:2010:RSL:1860134.1860346,Erwig:2009:SES:1608570.1608694,Erwig:2005:AGM:1062455.1062494}. For
example, XeLda checks if formulas process values with incorrect units
or if derived units clash with unit annotations. There also has been
considerable work on testing tools for
spreadsheets~\cite{fisher2006scaling,rothermel1998you,rothermel2001methodology,Carver:2006:EET:1159733.1159775}.

This work is complementary and orthogonal to \checkcell{}, which
works with standard, unannotated spreadsheets and focuses on unusual
interactions of data with formulas.


% What is Jaffry et al\. ~\cite{DBLP:journals/corr/abs-0803-1748}?

\subsection*{Statistical Outlier Analysis}

Techniques to locate outliers date to the earliest days of statistics,
when they were developed to make nautical measurements more
robust. Widely-used approaches include Peirce's criterion, Chauvenet's
criterion, and Grubb's test for
outliers~\cite{barnett1994outliers}. All of these techniques require
that data belong to a known distribution, primarily the normal
(Gaussian). Unfortunately, input data does not necessarily fit any
statistical distribution. Moreover, identifying outliers leads to
false positives when they do not materially contribute to the result
of a computation (i.e., have no impact). By contrast, data debugging only reports
data items with a substantial impact on a computation.

%\checkcell{} leverages the
%fact that its statistical sampling approach forces observed effects to
%follow a Gaussian distribution (as Section~\ref{sec:FIXME} explains),
%allowing it to use the Peirce outlier test to determine whether a
%particular value has an unusual impact on a computation.

