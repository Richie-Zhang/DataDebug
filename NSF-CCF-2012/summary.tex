\projectname{}

\noindent{\bf\textsf{Intellectual Merit.}} 
Correctness is a primary
concern for most computations. Techniques to reduce program errors range from testing and
runtime assertions, to dynamic and static analysis tools that can
discover a wide range of bugs. However, a program is just part of
a computation: if its input contains errors, the result
is likely to not be correct. Unlike programs, data
cannot easily be tested or analyzed for correctness.  Part of the
problem is that it is difficult to decide whether data
is wrong. For example, the number \texttt{1234}
might be correct, or the correct value might
be \texttt{12.34}. Typographical errors can change data items by
orders of magnitude. Unfortunately, finding this kind of mistake via
manual data auditing is onerous, unscalable, and error-prone. Data
errors can be costly: errors in spreadsheet data have led to losses of
millions of dollars.

This project proposes \emph{\bf data debugging}, an approach for
locating likely data errors. Since it is impossible to know \emph{a
priori} whether data are erroneous or not, data debugging aims to do
the next best thing: \emph{locating data that has an unusual impact on
the computation}. Intuitively, data that has an inordinate impact on
the result of a computation is either very important, or it is wrong. By
contrast, wrong data whose presence has no particularly unusual effect
on the final result does not merit special attention.

Data debugging combines data dependence analysis and statistical
analysis to find and rank data based on the unusualness of its impact
on the results of a computation. Data debugging works by first
building a data dependence graph of the computations. It then measures
data impact by replacing data items with data chosen from the same
group (such as a range in a spreadsheet formula) and observing the
resulting changes in computations that depend on that data. This
non-parametric approach allows data debugging to find errors in both
numeric and non-numeric data, without any requirement that data follow
any particular statistical distribution.

By calling attention to data with unusual impact, data debugging can
provide insights into both the data and the computation and reveal
errors. Data debugging is especially well-suited for data-intensive
programming environments like databases and spreadsheets that
intertwine data and programs, e.g., with queries and formulas.

We have developed a prototype data debugging tool
for spreadsheets. This tool, called
\checkcell{},
 highlights data whose impact crosses a threshold of unusualness,
ranking these cells by coloring them in shades proportionally to their
impact: the brighter a cell is highlighted, the more unusual impact it
has.  While untuned, our prototype is empirically and analytically
efficient and effective; analysis time ranges from seconds to minutes.
A case study employing human workers via a crowdsourcing platform
verifies that \checkcell{} finds actual data entry errors.

This project will develop data debugging in several key directions,
broadening its scope and improving its
performance: (1) extension of data debugging to database systems and
MapReduce tasks; (2) static analysis of formulas and queries to enable
optimizations; (3) derivation of incremental versions of data
debugging algorithms to minimize recomputations when data changes; and (4)
parallelizing data debugging to take advantage of multicore
architectures. This work will make it possible for data debugging to
detect likely errors as soon as they are entered.

\smallskip
\noindent{\bf\textsf{Broader Impact.}}
Data debugging is a new paradigm for attacking the problem of errors
in data by leveraging the interaction between data and the programs
that operate on them.  If successful, this project will dramatically
reduce the risks of human data entry errors or data corruption,
increasing the reliability of computations over data, and potentially
saving companies millions of dollars.

The investigators will make their benchmarks and tools publicly
available, adding to the national research infrastructure. Several of
the investigators' prior tools and systems are widely used in research
and industry, including the Hoard high-performance memory manager, and the
DieHard error-tolerant and DieHarder secure runtime systems now
incorporated in Windows. Educational impact will include training
graduate and undergraduate students, contributing to the technology
workforce, and outreach to under-represented groups via the inclusion
of female students from nearby Mount Holyoke and Smith Colleges.

\smallskip
\noindent{\bf\textsf{Key words:}} Programming Languages; Databases; Big Data; Errors
